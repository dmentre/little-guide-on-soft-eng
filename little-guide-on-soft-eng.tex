\documentclass[a4paper]{article}
\usepackage{hyperref}
\usepackage[utf8]{inputenc}

%% rule macro
\newcounter{RuleNumber}
\setcounter{RuleNumber}{1}
\newcommand{\Rule}{Rule \theRuleNumber:\stepcounter{RuleNumber}}

\title{Little Guide on Software Engineering}
\author{David \textsc{Mentré}\\
  \href{mailto:david.mentre@bentobako.org}{david.mentre@bentobako.org}\\
  Version 0.4}
\date{2016-05-21}

\begin{document}
\maketitle

%% TO DO %%

The purpose of this guide is to give in a concise manner the essential
rules of good Software Engineering.

The following rules are given in order, most important first.

\begin{description}

\item[\Rule{} Make the right software.]

  From the beginning consult final users. Show them early designs or
  prototypes and take into account their feedback. The software is
  developed for them.\cite[sec 10, 11 \& 45]{andrew99}

\item[\Rule{} Code for humans, not machines.]

  Use meaningful names: your reader should understand the purpose of
  variables, procedures, methods, packages, ... without having to look
  at its context or internals.\cite[sec 1.1]{kernighan99} More
  generally, make your intent the clearest possible. Develop for others.

\item[\Rule{} Make the software right: test or prove thoroughly your code.]

  Do Unit, Integration, Validation and verification, Resource
  exhaustion, errors, and recovery, Performance and Usability
  tests.\cite[sec 34 \& 43]{andrew99}\cite[chap 6]{kernighan99}
  Mathematically prove the correctness of the most critical parts, they
  cannot fail.

\item[\Rule{} Be DRY: Don't Repeat Yourself.]

  Say something only once. If you repeat a piece in a similar way,
  factorize the common parts. Only have one reference for any piece of
  data, preferably in text form.\cite[sec 7]{andrew99}

\item[\Rule{} Cut program in orthogonal, loosely coupled modules.]

  To manage complexity, use features of your programming language to cut
  your code in classes, packages, etc. Each module should have a minimal
  interface revealed to other modules that masks its own ``secret''. A
  change in one module should not impact
  others.\cite{parnas1972}\cite[chap 3]{meyer1997}

\item[\Rule{} Document the why in code and the how of interfaces.]

  Use comments to explain the ``why'' of your code, the non obvious
  things. The ``how'' is the code.\cite[sec 44]{andrew99} However, for
  interfaces, it is important to clearly explain your users how to use
  them, with the most simplest examples.\cite[chap 4]{kernighan99}

\item[\Rule{} Use version control system and Continuous Integration.]

  Version control systems (git, Subversion, ...) free your mind,
  allowing to go back at any point in your development an issue would
  occur.\cite[sec 17]{andrew99} By using Continuous Integration, you
  build, test, prove and release each change. You'll discover issues
  sooner, you'll be ready at any time.

\item[\Rule{} Specify in advance but use iterative development.]

  Specifications allow to think about the software. Think hard at things
  you won't change easily later. Define used vocabulary in a dictionary,
  to share a common understanding. However, whatever plan you'll make,
  it is going to change. So start with an iterative approach, being
  prepared to update all software development artifacts.

\item[\Rule{} Automate everything.]

  Don't enter manual commands, use scripts to automatically build, test,
  prove, deploy, ... your code. Such scripts are reproducible and
  precisely encode project knowledge.\cite[sec 42]{andrew99}

\item[\Rule{} Use contracts and assertions.]

  Contracts and assertions are like executable comments: they document
  your code but they break if not satisfied and are never
  out-of-date.\cite[sec 21]{andrew99}\cite[rule
    5]{holzmann2006}\cite[chap 11]{meyer1997}

\item[\Rule{} Use minimal dependencies but don't reinvent the wheel.]

  External dependencies can make your life much easier, you reuse the
  experience of others. But too many dependencies rapidly become
  unmanageable, with bugs and security issues you don't understand.

\end{description}

\paragraph{Please comment} If you have some feedback, disagree or agree
with above rules, don't understand them or have suggestions, let me
know.

\paragraph{Acknowledgments} Many thanks to Thomas Genet and Yann
Régis-Gianas for reviewing draft of this document. Errors are mine.

\paragraph{License CC0} To the extent possible under law, David Mentré
has waived all copyright and related or neighboring rights to this
document. This work is published from France.

\bibliographystyle{plain}
\bibliography{bib-lgose}

\end{document}
