\documentclass[a4paper]{article}
\usepackage{hyperref}
\usepackage[utf8]{inputenc}

%% rule macro
\newcounter{RuleNumber}
\setcounter{RuleNumber}{1}
\newcommand{\Rule}{Rule \theRuleNumber:\stepcounter{RuleNumber}}

\title{Little guide on Software Engineering}
\author{David Mentré\\\url{david.mentre@bentobako.org}\\Version 0.1}
\date{2016-05-16}

\begin{document}
\maketitle

%% TO DO %%

% Add references

The purpose of this guide is to give in a concise manner the essential
rules of good Software Engineering.

The following rules are given in order, most important first.

\begin{description}
\item[\Rule{} Use meaningful names]

 Names of variables, procedures, methods, packages, ... should let the
 reader understand the purpose of the entity without having to look at
 its context or internals.

\item[\Rule{} Test or prove thoroughly your code]

  Do Unit, Integration, Validation and verification, Resource
  exhaustion, errors, and recovery, Performance and Usability
  tests. Mathematically prove the correctness of the most critical
  parts, no bug should be allowed to slip in them.

\item[\Rule{} Be DRY: Don't Repeat Yourself]

  Say something only once. If you repeat a piece in a similar way,
  factorize the common parts in a function. Only have one reference for
  any piece of data, preferably in text form.

\item[\Rule{} Cut program into orthogonal modules with loose dependency
  between them]

  To manage complexity, you should use features of your programming
  language to cut your code in classes, packages, etc. Each module
  should have a minimal interface revealed to other modules that masks
  its own ``secret''.

\item[\Rule{} Document the why in code and the how of interfaces]

  Use comments to explain the ``why'' of your code, the non obvious
  things. The ``how'' is the code. However, for interfaces, it is
  important to clearly explain your users how to use them, with the most
  simplest examples.

\item[\Rule{} Use version control system and Continuous Integration]

  Version control system (git, Subversion, ...) free your mind, allowing
  to go back at any point in your development an issue would occur. By
  using a Continuous Integration system that pulls out sources from
  version control, built and test them up to release state, you'll
  discover issues sooner. Sooner is better.

\item[\Rule{} Automate everything]

  Don't enter manual commands, use scripts to automatically build, test,
  prove, deploy, ... your code. Such scripts are reproducible and
  precisely encode knowledge.

\item[\Rule{} Specify but use iterative development]

  Specifications allow to think about the software, its requirements,
  its architecture and design, etc. However, whatever plan you'll make,
  it is going to change. So start from the beginning with an iterative
  approach, being prepared to change all the software development
  artifacts.

\item[\Rule{} Use contracts and assertions]

  Contracts and assertions are like executable comments: they document
  your code but they break if not satisfied and are never out-of-date.

\item[\Rule{} Use minimal dependencies but don't reinvent the wheel]

  External dependencies can make your life much easier, you reuse the
  experience of others. But too many dependencies migth become
  unmanageable, with bugs and security issues you don't understand.

\end{description}

\paragraph{Please comment} If you have some feedback, disagree or agree
with above rules, don't understand them or have suggestions, let me
know.

\end{document}
